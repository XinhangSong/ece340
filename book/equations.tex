\documentclass{article}

\usepackage{geometry}
\geometry{a4paper, total={190mm,257mm}, left=10mm, top=10mm}

\usepackage{amsmath}

\begin{document}


\section{Chapter 1}
\begin{itemize}
\item Energy E of a photon of light in eV: $$\lambda = \frac{1.24eV}{E}$$
\item Distance D between adjacent planes in cubic lattices: $$D = \frac{a}{\sqrt{ h^2 + k^2 + l^2} }$$
\item Angle between 2 Miller index directions A and B: $$\cos(\theta) = \frac{A\bullet B}{|A| |B|}$$
\end{itemize}

\section{Chapter 2}
\begin{itemize}
\item Planck relationship: $$ E=hv = (\frac{h}{2 \pi})(2 \pi v) = \hbar \omega$$
\item Classical energy of a particle: $$\frac{1}{2}mv^2 = \frac{1}{2}\frac{m^2v^2}{m}=\frac{\rho^2}{2m}$$
\item De Broglie: $$\lambda = \frac{h}{\rho} = \frac{h}{mv}\Rightarrow\rho=\frac{h}{\lambda}=\frac{h}{2\pi}\frac{2\pi}{\lambda}=\hbar k$$
\item Momentum or energy in terms of k can be derived by combining De Broglie with the classical energy of a particle:$$E=\hbar\omega=\frac{\rho^2}{2m}=\frac{\hbar^2 k^2}{2m}$$
\item Rydberg Constant: $$R=109,678cm^{-1}$$
\item Lyman: $$v=cR(\frac{1}{1^2}-\frac{1}{n^2}), n=2,3,4,...$$
\item Balmer: $$v=cR(\frac{1}{2^2}-\frac{1}{n^2}),n=3,4,5...$$
\item Paschen: $$v=cR(\frac{1}{3^2}-\frac{1}{n^2}),n=4,5,6,...$$
\item Postulate for Bohr:$$\rho_0=n\hbar,n=1,2,3,4,...$$
\item Finding radial forces on orbiting electron:\\(Electrical force toward nucleus) = (Equivalent force in terms of radial acceleration)\\$$-\frac{q^2}{kr^2}=-\frac{mv^2}{r}$$
  $$\rho_0=n\hbar=mvr  \Rightarrow mv^2=\frac{m^2v^2}{m}=\frac{n^2\hbar^2}{mr^2}$$ $$\frac{q^2}{kr^2}=\frac{1}{mr}\frac{n^2\hbar^2}{r^2}\Rightarrow r_n=\frac{kn^2\hbar^2}{mq^2}$$ $|r_n$ is the radius of the nth orbit
  $$-\frac{q^2}{kr^2}=-\frac{mv^2}{r}\Rightarrow \frac{n\hbar}{rm}\Rightarrow=\frac{q^2}{kn\hbar}$$by subbing r from above.  \\\\$\Rightarrow$ K.E. of $e^-=$$$\frac{1}{2}mv^2=\frac{mq^4}{2k^2n^2\hbar^2}$$\\\\P.E. of $e^-=$$$-\frac{q^2}{kr_n}=-\frac{mq^4}{k^2n^2\hbar^2}$$ by subbing r\\\\Total Energy of $e^-=$$$E_n=KE=PE=-\frac{mq^4}{2k^2n^2\hbar^2}=-KE$$
\item Energy difference between orbits: $$E_{n2}-E_{n1}=\frac{mq^4}{2k^2\hbar^2}(\frac{1}{n_1^2}-\frac{1}{n_2^2})$$
\item Frequency of light given by a transition between orbits:$$V_{21}=\left[\frac{mq^4}{2k^2\hbar^2h}\right](\frac{1}{n_1^2}-\frac{1}{n_2^2})$$
\item Heisenberg uncertainty principle:\\
  \begin{equation}
    \label{Heisenberg}
    (\Delta x)(\Delta \rho_x) \ge\frac{\hbar}{2}\\
  \end{equation}
  \begin{equation}
    \label{Heisenberg part 2}
    (\Delta E)(\Delta t)\ge\frac{\hbar}{2}
  \end{equation}
\end{itemize}
\subsection{Quantum Mechanics}
\begin{tabular}{lcr}
  Classical Variable & $\rightarrow$ & Quantum Operator\\
  \hline
  $x$ & $\rightarrow$ & $x$\\
  $f(x)$ & $\rightarrow$ & $f(x)$\\
  $\rho(x)$ &$ \rightarrow$ & $\frac{\hbar}{j} \frac{\partial}{\partial x}$\\
  E & $\rightarrow$ & $-\frac{\hbar}{j} \frac{\partial}{\partial t}$\\
  \hline
\end{tabular}\\
\begin{itemize}
\item Normalization of the probability density (the wave function is the probability density):
  \begin{equation}
    \label{Normalization_of_the_wave_function}
    \int_{-\infty}^{\infty} \Psi^{*}\Psi dxdydz=1
  \end{equation}
\item Time averaged expectation of the particle state
  \begin{equation}
    \label{Time_average_of_the_state}
    <Q>=\int_{-\infty}^{\infty}\Psi^{*}Q_{op}\Psi dxdydz
  \end{equation}
\item Classical energy of a partical:\begin{equation} \label{classical_energy_of_a_partical}
  KE+PE=E\Rightarrow\frac{1}{2}mv^2 + V = E\end{equation}
  \begin{equation}
    \label{part_2}
    \frac{1}{2}mv^2=\frac{1}{2}\frac{(mv)^2}{m}=\frac{1}{2}\frac{\rho^2}{m}=
    \frac{\rho^2}{2m}\Rightarrow\frac{\rho^2}{2m}+V=E
  \end{equation}
  $$\rho\rightarrow\frac{\hbar}{j}\frac{\partial}{\partial x},E\rightarrow -\frac{\hbar}{j}\frac{\partial}{\partial t}$$$$\Rightarrow \frac{-1}{2m}\hbar\frac{\partial^2}{\partial x^2}\Psi(x,t)+V(x)\Psi(x,t)=\frac{-\hbar}{j}\frac{\partial\Psi(x,t)}{\partial t}$$ \\ 
  where $$(\frac{\partial}{\partial x})^2\rightarrow\frac{\partial^2}{\partial x^2}, j^2 = -1$$
\item Wave function in 3D then:
  \begin{equation}\label{3D_wave_function}
    \frac{-\hbar}{2m}\nabla^2\Psi+V\Psi=\frac{-\hbar}{j}\frac{\partial\Psi}{\partial t} \ni \nabla^2\Psi=\frac{\partial^2\Psi}{\partial x^2}+\frac{\partial^2\Psi}{\partial y^2}+\frac{\partial^2\Psi}{\partial z^2}\end{equation}
\item Separation of variables:
  \begin{equation}\label{separated}
    \frac{-\hbar}{2m}\frac{\partial^2\Psi(x,t)}{\partial x^2}+V(x)\Psi(x,t)=\frac{-\hbar}{j}\frac{\partial\Psi(x,t)}{\partial t}
  \end{equation}
  $$\Rightarrow$$
  \begin{equation}\label{still_separated}
    -\frac{\hbar}{2m}\frac{\partial^2\psi(x)}{\partial x^2}\phi(t)+V(x)\psi(x)\phi(t)=-\frac{\hbar}{j}\psi(x)\frac{\partial\phi}{\partial t}
  \end{equation}
  $$\Rightarrow$$
  \begin{equation}\label{time_dependent}
    \frac{d\phi(t)}{dt}+\frac{j}{\hbar}E\phi(t)=0	
  \end{equation} (time dependent portion)
  \begin{equation}\label{time_independent}
    -\frac{\hbar}{2m}\frac{d^2\psi(x)}{dx^2}+V(x)\psi(x)=E\psi(x)
  \end{equation} (time independent portion)
  
\item E $\equiv$ equivalent constant, corresponds to total energy of the particle
\item Wave function as linear combination of various eigenfunctions
  \begin{equation}
    \label{eigenfunctions}
    \psi (x,t)=\sum_{n}C_{n}\Psi_{n}e^{-j\frac{E_{n}}{\hbar t}} \ni E_{n}\equiv nth~prefactor
  \end{equation}
\item Infinite potential well
  \[V(x) = \left\{
  \begin{array}{lr}
    0 & , x \neq 0 \text{ and } x \neq L\\
    \infty & , x = 0 \text{ or } x = L
  \end{array}
  \right.
  \]
  $$\Rightarrow$$
  \begin{equation}
    (\frac{\hbar^2}{2m}\frac{d^2}{dx^2}+V(x))\psi(x)=E\psi(x)\Rightarrow\frac{d^2\psi(x)}{dx^2}=\frac{2m}{\hbar}E\psi(x)
  \end{equation}$$\Rightarrow$$
  \begin{equation}
    \frac{d^2\psi(x)}{dx^2}+\frac{2m}{\hbar^2}E\psi(x)=0
  \end{equation}$$\Rightarrow$$
  \begin{equation}
    \psi(x)=\{\sin(kx),\cos(kx)\}\ni k=\frac{\sqrt{2mE}}{\hbar}
  \end{equation}
  \begin{equation}
    \cos(kx)\neq 0\text{ when }x=0\Rightarrow A\sin(kx)\text{ is our solution. }\ni k=\frac{\sqrt{2mE}}{\hbar}
  \end{equation}
  \begin{equation}
    \text{when potential is 0 at x = 0 and L, }k=\frac{n\pi}{L},n=\{1,2,3,...\}
  \end{equation}
  \begin{equation}
    \text{therefore: }\frac{\sqrt{2mE_n}}{\hbar} = \frac{n\pi}{L}\Rightarrow E_n=\frac{n^2\pi^2\hbar^2}{wmL^2}
  \end{equation}
\item A is found by normalizing the probability density integral
  \begin{equation}
    \int_{-\infty}^{\infty}\Psi^*\Psi dx=\int_0^LA^2\sin^2(\frac{n\pi x}{L})dx = A^2\frac{L}{2} \text{\\review~trig~ calc~ and~ show~  the~ process~ here}
  \end{equation}
  \\Set the above to 1 to find A
  \begin{equation}
    \frac{A^2L}{2}=1\Rightarrow A=\sqrt{\frac{2}{L}}
  \end{equation}
  \begin{equation}
    \text{Therefore:~ }\psi=\sqrt{\frac{2}{L}}\sin(\frac{n\pi}{L})\text{ ~for~ an~ infinite~ well}
  \end{equation}
\item Parabolic Potential Well (simple harmonic oscillator)
  \begin{equation}
    V(x)=kx^2, E_n=(n+\frac{1}{2})\hbar\omega
  \end{equation}
\item Coulombic Potential Well
  \begin{equation}
    \frac{d^2}{d\phi^2}\Phi+m^2\Phi=0\Rightarrow\Phi_m(\phi)=Ae^{jm\phi}
  \end{equation}
  \begin{equation}
    1=\int_0^{2\pi}\Psi^*\Psi d\psi\Rightarrow\int_0^{2\pi}\Phi_m^*(\phi)\Phi_m(\phi)d\phi=1
  \end{equation}  \begin{equation}
    \Rightarrow A^2\int_0^{2\pi}e^{-jm\phi}e^{jm\phi}d\phi
  \end{equation}
  $$\Rightarrow$$
  \begin{equation}A^2\int_0^{2\pi}d\phi=2\pi A^2
  \end{equation}$$\Rightarrow$$
  \begin{equation}A=\frac{1}{\sqrt{2\pi}}\Rightarrow \Phi_m(\phi)=\frac{1}{\sqrt{2\pi}}e^{jm\phi}
  \end{equation}$$\ni m=\{...,-3,-2,-1,0,1,...\}$$\\
  similar for $\Theta_{l}(\theta)$ and $R_n(r)$
\item Atomic numbers describing allowable states in a hydrogen atom\\
  $n = 1,2,3,...$\\$l=0,1,2,3,...,n-1$\\$m (m_l) = -l,...,-2,-1,0,1,2,...,l$\\$s (m_s) = \frac{\pm\hbar}{2}$\\
\item \begin{tabular}{c|c|c|c|c|c}
  n & l & m & $\frac{s}{\hbar}$& Allowable states in subshell & Allowable states in complete shell \\
  \hline
  1 & 0 & 0 & $\pm\frac{1}{2}$ & 2 & 2 \\
  2 & 0 & 0 & $\pm\frac{1}{2}$ & 2 &   \\
  2 & 1 & -1 & $\pm\frac{1}{2}$ &   & 8 \\
  2 & 1 & 0 & $\pm\frac{1}{2}$ & 6  &   \\
  2 & 1 & 1 & $\pm\frac{1}{2}$&   &   \\
  \hline
  3 & 0 & 0 & $\pm\frac{1}{2}$ & 2  & 18  \\
  \hline
  3 & 1 & -1 & $\pm\frac{1}{2}$ &   &   \\
  3 & 1 & 0 & $\pm\frac{1}{2}$ & 6  & 18  \\
  3 & 1 & 1 & $\pm\frac{1}{2}$ &   &   \\
  \hline
  3 & 2 & -2 & $\pm\frac{1}{2}$ &   &   \\
  3 & 2 & -1 & $\pm\frac{1}{2}$ &   &   \\
  3 & 2 & 0 & $\pm\frac{1}{2}$ & 10  & 18  \\
  3 & 2 & 1 & $\pm\frac{1}{2}$ &   &   \\
  3 & 2 & 2 & $\pm\frac{1}{2}$ &   &   \\    
\end{tabular}
\item n is the principle atomic number
\item l is the number that determines s,p,d,f,g,...
\end{itemize}

\section{Chapter 3}
\begin{itemize}
\item Equilibrium number of EHP's in pure Si at room temp: $$10^{10}\frac{EHP}{cm^3}$$
\item Si atom density in pure Si at room temp: $$5*10^{22}\frac{atoms}{cm^3}$$
\item Releationship of (E,k) for a free electron related to electron mass
  \begin{equation}
    \frac{d^{2}E}{dk^2}=\frac{\hbar^2}{m}\Rightarrow m^*=\frac{\hbar^2}{\frac{d^2E}{dk^2}}\text{~so~the~curvature~of~the~band~determines~the~}e^{-}\text{~effective~mass}
  \end{equation}
\item Velocity of an electron (v) is the group velocity of a quantum mechanical electron wavepacket
  \begin{equation}v=\frac{d\omega}{dk} = (\frac{1}{\hbar})\frac{dE}{dk}\end{equation}
\item Rate of recombination of electrons and holes $r_i$ is proportional to equilibrium concentrations of electrons and holes $n_0$ and $p_0$
  \begin{equation}r_i=\alpha_rn_0p_0=\alpha_rn_i^2=g_i\end{equation}
\item Calculating the approximate energy required to excite a donor electron into conduction band.  \\
  
  Consider the loosely bound electron as circling the tightly bound ``core'' electons.

  Magnitude of the ``ground state'' is:
  \begin{equation}E=\frac{mq^4}{2k^2\hbar^2}\ni k=4\pi\epsilon_0\epsilon_r, m\equiv m_n^*\text{~(conductivity~effective~mass)}\end{equation}
\item Radius of of electron orbit around donor, assuming ground state.  Equation was pulled from chapter 2
  \begin{equation}r=\frac{4\pi\epsilon_r\epsilon_0\hbar^2}{m_n^*q^2}\end{equation} 
\item Intrinsic carrier concentration $n_i$ of Si is about $10^{10}cm^{-3}$ at room temperature
\end{itemize}
\subsection{Carrier Concentrations}
\begin{itemize}
\item Standard Fermi Equation:
  \begin{equation}f(E)=\frac{1}{1+e^{\frac{(E-E_F)}{kT}}}\ni\text{k~is~the~Boltzman~constant~} k=8.62*10^{-5}\frac{eV}{K}=1.38*10^{-23}\frac{J}{K}\end{equation}
\item Calculate the number of electrons and holes in a semiconductor when you know the density of states in valence and conduction bands are known
  \begin{itemize}
  \item Concentration of electrons in the conduction band by integral:
    \begin{equation}n_0=\int_{E_c}^{\infty}f(E)N(E)dE\ni N(E)d(E)\text{~is~the~density~of~states~}(cm^{-3})\text{~in~the~energy~range~of~dE}\end{equation}
  \item Concentration of electrons in the conduction band by using \textit{Effective Density of states}:
    \begin{equation} n_0 = N_c f (E_c) \ni N_c \text{~is~the~effective~density~of~states~at~} E_c \end{equation}
  \item When E-$E_F$ is several kT we can simplify the equation by recognizing that the exponential dominates the denominator
    \begin{equation}f(E_c)=\frac{1}{1+e^{\frac{(E-E_F)}{kT}}}\Rightarrow e^{\frac{(E-E_F)}{kT}}\end{equation}
    \begin{equation}n_0=N_ce^{-\frac{E_c-E_F}{kT}}\end{equation}
  \item Effective Density of States is:
    \begin{equation}N_c=2(\frac{2\pi m_n^*kT}{h^2})^{\frac{3}{2}}\ni m_n^*\equiv \text{~the~density~of~states~effective~mass~for~electrons}\end{equation}
  \item Density of States Effective Mass for Electrons is:
    \begin{equation}(m_n^*)^{\frac{3}{2}}=6(m_.m_t^2)^{\frac{1}{2}}\end{equation}
  \item So by similar reasoning, the concentration of holes in the valence band is$:$
    \begin{equation}p_0 = N_v[1 - f(E_v)]\end{equation}
      where $N_v$ is the effective density of states for the valence band.
    \item The probability of finding an empty state at $E_v$ is$:$
      \begin{equation}1 - f(E_v) = 1 - \frac{1}{1 + e^{\frac{E_V - E_F}{kT}}} \simeq e^{-\frac{(E_F - E_v)}{kT}} \end{equation}
    \item So the concentration of holes in the valence band is$:$
      \begin{equation}p_0 = N_ve^{-\frac{E_F-E_v}{kT}} \end{equation}
    \item With effective density of states in the valence band$:$
      \begin{equation}N_v = 2(\frac{2\pi m_p^{*}kT}{h^2})^{\frac{3}{2}} \end{equation}
    \item For \textit{intrinsic material}, $E_F$ is at some intrinsic level $E_i$ near the middle of the band gap.  If $N_v = N_c$ then it actually is the middle of the band gap.  Usually there is a difference in effective masses, though so there is usually a difference.  
    \item Intrinsic electron and hole concentrations are$:$
      \begin{equation} n_i = N_ce^{-\frac{E_c - E_i}{kT}}, p_i = N_ve^{-\frac{E_i - E_v}{kT}} \end{equation}
    \item The product of $n_0$ and $p_0$ at equilibrium is a constant for a particular material and temperature, even if the doping is varied
      \begin{equation} n_0p_0 = (N_ce^{-\frac{E_c - E_i}{kT}})(N_ve^{-\frac{E_i - E_v}{kT}}) = N_cN_ve^{-\frac{E_c - E_v}{kT}}= N_cN_ve^{-\frac{E_g}{kT}}\end{equation}
      \begin{equation} n_ip_i =  (N_ce^{-\frac{E_c - E_i}{kT}})(N_ve^{-\frac{E_i - E_v}{kT}}) = N_cN_ve^{-\frac{E_g}{kT}} \end{equation}
      \begin{align}\boxed{n_i=p_i \Rightarrow \sqrt{N_cN_v}e^{-\frac{E_g}{2kT}},~~ n_0p_0  = n_i^2} \end{align}
      \begin{align}
        \boxed{n_0 = n_ie^{-\frac{E_F - E_i}{kT}}} \\ \boxed{p_0 = n_ie^{-\frac{E_i - E_F}{kT}}} 
      \end{align}
  \end{itemize}
\end{itemize}

\subsection{Temperature Dependence of Carrier Concentrations}
The exponential temperature dependence on the intrinsic carrier concentration dominates the other factors$:$
\begin{equation}n_i(T)=2(\frac{2\pi kT}{h^2})^{\frac{3}{2}} (m_n^*m_p^*)^{\frac{3}{4}}e^{-\frac{E_g}{2kT}} \end{equation}
\subsection{Compensation and Space Charge Neutrality}
\begin{itemize}
\item Compensation changes the effective number of donors$:$
  \begin{equation}N_{d_e}= N_d - N_a \text{when~the number of donors exceeds the number of acceptors}\end{equation}
  \begin{equation}N_{a_e}= N_a - N_d \text{when~the number of acceptors exceeds the number of donors}\end{equation}
\item Space Charge Neutrality dictates that the number of positive charges (holes and ionized donor atoms) must balance the number of negative charges (electrons and ionized acceptor atoms)$:$
  \begin{equation}p_0 + N_d^+ = n_0 + N_a^- \Rightarrow n_0 = p_0 + (N_d^+ - N_a^-)\end{equation}
\item For a material that is doped n-type and the material is in the extrinsic region, we can approximate by recognizing that the donors dominate$:$
\begin{equation} n_0 \simeq N_d - N_a\end{equation}
\end{itemize}

\section{Drift of Carriers in Electric and Magnetic Fields}
\subsection{Conductivity and Mobility}


\end{document}
