\documentclass{article}
\begin{document}


\section{Chapter 1}
\begin{itemize}
\item Energy E of a photon of light in eV: $$\lambda = \frac{1.24eV}{E}$$
\item Distance D between adjacent planes in cubic lattices: $$D = \frac{a}{\sqrt{ h^2 + k^2 + l^2} }$$
\item Angle between 2 Miller index directions A and B: $$\cos\(\theta\) = \frac{A\bullet B}{|A| |B|}$$
\end{itemize}
\endsection
\section{Chapter 2}
\begin{itemize}
\item Planck relationship: $$ E=hv = (\frac{h}{2 \pi})(2 \pi v) = \hbar \omega$$
\item Classical energy of a particle: $$\frac{1}{2}mv^2 = \frac{1}{2}\frac{m^2v^2}{m}=\frac{\rho^2}{2m}$$
\item De Broglie: $$\lambda = \frac{h}{\rho} = \frac{h}{mv}\Rightarrow\rho=\frac{h}{\lambda}=\frac{h}{2\pi}\frac{2\pi}{\lambda}=\hbar k$$
\item Momentum or energy in terms of k can be derived by combining De Broglie with the classical energy of a particle:$$E=\hbar\omega=\frac{\rho^2}{2m}=\frac{\hbar^2 k^2}{2m}$$
\item Rydberg Constant: $$R=109,678cm^{-1}$$
\item Lyman: $$v=cR(\frac{1}{1^2}-\frac{1}{n^2}), n=2,3,4,...$$
\item Balmer: $$v=cR(\frac{1}{2^2}-\frac{1}{n^2}),n=3,4,5...$$
\item Paschen: $$v=cR(\frac{1}{3^2}-\frac{1}{n^2}),n=4,5,6,...$$
\item Postulate for Bohr:$$\rho_0=n\hbar,n=1,2,3,4,...$$
\item Finding radial forces on orbiting electron:\\(Electrical force toward nucleus) = (Equivalent force in terms of radial acceleration)\\$$-\frac{q^2}{kr^2}=-\frac{mv^2}{r}$$
$$\rho_0=n\hbar=mvr  \Rightarrow mv^2=\frac{m^2v^2}{m}=\frac{n^2\hbar^2}{mr^2}$$ $$\frac{q^2}{kr^2}=\frac{1}{mr}\frac{n^2\hbar^2}{r^2}\Rightarrow r_n=\frac{kn^2\hbar^2}{mq^2}$$ $|r_n$ is the radius of the nth orbit
$$-\frac{q^2}{kr^2}=-\frac{mv^2}{r}\Rightarrow \frac{n\hbar}{rm}\Rightarrowv=\frac{q^2}{kn\hbar}$$by subbing r from above.  \\\\$\Rightarrow$ K.E. of $e^-=$$$\frac{1}{2}mv^2=\frac{mq^4}{2k^2n^2\hbar^2}$$\\\\P.E. of $e^-=$$$-\frac{q^2}{kr_n}=-\frac{mq^4}{k^2n^2\hbar^2}$$ by subbing r\\\\Total Energy of $e^-=$$$E_n=KE=PE=-\frac{mq^4}{2k^2n^2\hbar^2}=-KE$$
\item Energy difference between orbits: $$E_{n2}-E_{n1}=\frac{mq^4}{2k^2\hbar^2}(\frac{1}{n_1^2}-\frac{1}{n_2^2})$$
\item Frequency of light given by a transition between orbits:$$V_{21}=\left[\frac{mq^4}{2k^2\hbar^2h}\right](\frac{1}{n_1^2}-\frac{1}{n_2^2})$$
\item Heisenberg uncertainty principle:\\
\begin{align}
\label{Heisenberg}
&(\Delta x)&(\Delta \rho_x)& \ge&\frac{\hbar}{2}\\
\label{Heisenberg part 2}
&(\Delta E)&(\Delta t)&\ge&\frac{\hbar}{2}
\end{align}
\end{itemize}
\subsection{Quantum Mechanics}
\begin{tabular}{lcr}
Classical Variable & \rightarrow & Quantum Operator\\
\hline
$x$ & \rightarrow & $x$\\
$f(x)$ & \rightarrow & $f(x)$\\
$\rho(x)$ & \rightarrow & $\frac{\hbar}{j} \frac{\partial}{\partial x}$\\
E & \rightarrow & $-\frac{\hbar}{j} \frac{\partial}{\partial t}$\\
\hline
\end{tabular}\\
\begin{itemize}
\item Normalization of the probability density (the wave function is the probability density):
	\begin{equation}
		\label{Normalization_of_the_wave_function}
		\int_{-\infty}^{\infty} \Psi^{*}\Psi dxdydz=1
	\end{equation}
\item Time averaged expectation of the particle state
	\begin{equation}
		\label{Time_average_of_the_state}
		<Q>=\int_{-\infty}^{\infty}\Psi^{*}Q_{op}\Psi dxdydz
	\end{equation}
\item Classical energy of a partical:\begin{equation} \label{classical_energy_of_a_partical}
	KE+PE=E\Rightarrow\frac{1}{2}mv^2 + V = E\end{equations}
	\begin{equation}
		\label{part_2}
	\frac{1}{2}mv^2=\frac{1}{2}\frac{(mv)^2}{m}=\frac{1}{2}\frac{\rho^2}{m}=\frac{\rho^2}{2m}\Rightarrow\frac{\rho^2}{2m}+V=E
\end{equation}
$$\rho\rightarrow\frac{\hbar}{j}\frac{\partial}{\partial x},E\rightarrow -\frac{\hbar}{j}\frac{\partial}{\partial t}$$$$\Rightarrow \frac{-1}{2m}\hbar\frac{\partial^2}{\partial x^2}\Psi(x,t)+V(x)\Psi(x,t)=\frac{-\hbar}{j}\frac{\partial\Psi(x,t)}{\partial t}$$ \\ 
where $$(\frac{\partial}{\partial x})^2\rightarrow\frac{\partial^2}{\partial x^2}, j^2 = -1$$
\item Wave function in 3D then:
	\begin{equation}\label{3D_wave_function}
	\frac{-\hbar}{2m}\nabla^2\Psi+V\Psi=\frac{-\hbar}{j}\frac{\partial\Psi}{\partial t} \ni \nabla^2\Psi=\frac{\partial^2\Psi}{\partial x^2}+\frac{\partial^2\Psi}{\partial y^2}+\frac{\partial^2\Psi}{\partial z^2}\end{equation}
\item Separation of variables:
	\begin{equation}\label{separated}
		\frac{-\hbar}{2m}\frac{\partial^2\Psi(x,t)}{\partial x^2}+V(x)\Psi(x,t)=\frac{-\hbar}{j}\frac{\partial\Psi(x,t)}{\partial t}
	\end{equation}
	$$\Rightarrow$$
	\begin{equation}\label{still_separated}
		-\frac{\hbar}{2m}\frac{\partial^2\psi(x)}{\partial x^2}\phi(t)+V(x)\psi(x)\phi(t)=-\frac{\hbar}{j}\psi(x)\frac{\partial\phi}{\partial t}
	\end{equation}
	$$\Rightarrow$$
\begin{equation}\label{time_dependent}
	\frac{d\phi(t)}{dt}+\frac{j}{\hbar}E\phi(t)=0	
	\end{equation} (time dependent portion)
	\begin{equation}\label{time_independent}
		-\frac{\hbar}{2m}\frac{d^2\psi(x)}{dx^2}+V(x)\psi(x)=E\psi(x)
	\end{equation} (time independent portion)

\item E $\equiv$ equivalent constant, corresponds to total energy of the particle
\item Wave function as linear combination of various eigenfunctions
	\begin{equation}
	\label{eigenfunctions}
	\psi(x,t)=\Sum_{n}C_{n}\Psi_{n}e^{-j\frac{E_{n}}{\hbar t}} \ni E_{n}\equiv nth~prefactor
	\end{equation}
\item Infinite potential well
		\[V(x) = \left\{
				  \begin{array}{lr}
					  0 & , x \neq 0 \text{ and } x \neq L\\
			\infty & , x = 0 \text{ or } x = L
					    \end{array}
					    \right.
				    \]
				    $$\Rightarrow$$
    \begin{equation}
	    (\frac{\hbar^2}{2m}\frac{d^2}{dx^2}+V(x))\psi(x)=E\psi(x)\Rightarrow\frac{d^2\psi(x)}{dx^2}=\frac{2m}{\hbar}E\psi(x)
    \end{equation}$$\Rightarrow$$
    \begin{equation}
	    \frac{d^2\psi(x)}{dx^2}+\frac{2m}{\hbar^2}E\psi(x)=0
    \end{equation}$$\Rightarrow$$
    \begin{equation}
	    \psi(x)=\{\sin(kx),\cos(kx)\}\ni k=\frac{\sqrt{2mE}}{\hbar}
    \end{equation}
\end{itemize}
\endsubsection
\endsection
\section{Chapter 3}
\begin{itemize}
\item Equilibrium number of EHP's in pure Si at room temp: $$10^{10}\frac{EHP}{cm^3}$$
\item Si atom density in pure Si at room temp: $$5*10^{22}\frac{atoms}{cm^3}$$
\end{itemize}
\endsection

\end{document}
